\documentclass[../main.tex]{subfiles}

\begin{document}
% You'll find mixed opinions about classifying your skills by proficiency. I don't do it in my resume.
% If you still want to do the classification, it's hard to assess what makes a novice/expert developer in a specific language. I'd group the languages / skills by how much time you've spent working with them (In years)
% Even though I used Java in college I don't list it because I'm not interested in working with it. I advise doing the same for technology that you're not interested in
% I don't remember exactly how I decided to add the next lines to my resume but might be worth the shot adding something similar (here or in a different section)
% Mentioned next lines
% Perseverance to solve challenging problems. Strong Google-fu / StackOverflow skills
% Quick learner, can digest new information at a fast pace, and have no problem adapting
% Self-motivated, I’m able to get things done without constant supervision
 
\emoji{thumbs-up} \textbf{Expert}: Python, Git, VSCode, Spanish

\emoji{ok-hand} \textbf{Proficient:} C++, Rust, ZSH

\emoji{pinching-hand} \textbf{Novice} Java, CSS, JavaScript, HTML, Data Analysis, R, SQL, VIM, Godot, C\#, Bash, Linux
\end{document}
